\documentclass[11pt]{article}

    \usepackage[breakable]{tcolorbox}
    \usepackage{parskip} % Stop auto-indenting (to mimic markdown behaviour)
    

    % Basic figure setup, for now with no caption control since it's done
    % automatically by Pandoc (which extracts ![](path) syntax from Markdown).
    \usepackage{graphicx}
    % Keep aspect ratio if custom image width or height is specified
    \setkeys{Gin}{keepaspectratio}
    % Maintain compatibility with old templates. Remove in nbconvert 6.0
    \let\Oldincludegraphics\includegraphics
    % Ensure that by default, figures have no caption (until we provide a
    % proper Figure object with a Caption API and a way to capture that
    % in the conversion process - todo).
    \usepackage{caption}
    \DeclareCaptionFormat{nocaption}{}
    \captionsetup{format=nocaption,aboveskip=0pt,belowskip=0pt}

    \usepackage{float}
    \floatplacement{figure}{H} % forces figures to be placed at the correct location
    \usepackage{xcolor} % Allow colors to be defined
    \usepackage{enumerate} % Needed for markdown enumerations to work
    \usepackage{geometry} % Used to adjust the document margins
    \usepackage{amsmath} % Equations
    \usepackage{amssymb} % Equations
    \usepackage{textcomp} % defines textquotesingle
    % Hack from http://tex.stackexchange.com/a/47451/13684:
    \AtBeginDocument{%
        \def\PYZsq{\textquotesingle}% Upright quotes in Pygmentized code
    }
    \usepackage{upquote} % Upright quotes for verbatim code
    \usepackage{eurosym} % defines \euro

    \usepackage{iftex}
    \ifPDFTeX
        \usepackage[T1]{fontenc}
        \IfFileExists{alphabeta.sty}{
              \usepackage{alphabeta}
          }{
              \usepackage[mathletters]{ucs}
              \usepackage[utf8x]{inputenc}
          }
    \else
        \usepackage{fontspec}
        \usepackage{unicode-math}
    \fi

    \usepackage{fancyvrb} % verbatim replacement that allows latex
    \usepackage{grffile} % extends the file name processing of package graphics
                         % to support a larger range
    \makeatletter % fix for old versions of grffile with XeLaTeX
    \@ifpackagelater{grffile}{2019/11/01}
    {
      % Do nothing on new versions
    }
    {
      \def\Gread@@xetex#1{%
        \IfFileExists{"\Gin@base".bb}%
        {\Gread@eps{\Gin@base.bb}}%
        {\Gread@@xetex@aux#1}%
      }
    }
    \makeatother
    \usepackage[Export]{adjustbox} % Used to constrain images to a maximum size
    \adjustboxset{max size={0.9\linewidth}{0.9\paperheight}}

    % The hyperref package gives us a pdf with properly built
    % internal navigation ('pdf bookmarks' for the table of contents,
    % internal cross-reference links, web links for URLs, etc.)
    \usepackage{hyperref}
    % The default LaTeX title has an obnoxious amount of whitespace. By default,
    % titling removes some of it. It also provides customization options.
    \usepackage{titling}
    \usepackage{longtable} % longtable support required by pandoc >1.10
    \usepackage{booktabs}  % table support for pandoc > 1.12.2
    \usepackage{array}     % table support for pandoc >= 2.11.3
    \usepackage{calc}      % table minipage width calculation for pandoc >= 2.11.1
    \usepackage[inline]{enumitem} % IRkernel/repr support (it uses the enumerate* environment)
    \usepackage[normalem]{ulem} % ulem is needed to support strikethroughs (\sout)
                                % normalem makes italics be italics, not underlines
    \usepackage{soul}      % strikethrough (\st) support for pandoc >= 3.0.0
    \usepackage{mathrsfs}
    

    
    % Colors for the hyperref package
    \definecolor{urlcolor}{rgb}{0,.145,.698}
    \definecolor{linkcolor}{rgb}{.71,0.21,0.01}
    \definecolor{citecolor}{rgb}{.12,.54,.11}

    % ANSI colors
    \definecolor{ansi-black}{HTML}{3E424D}
    \definecolor{ansi-black-intense}{HTML}{282C36}
    \definecolor{ansi-red}{HTML}{E75C58}
    \definecolor{ansi-red-intense}{HTML}{B22B31}
    \definecolor{ansi-green}{HTML}{00A250}
    \definecolor{ansi-green-intense}{HTML}{007427}
    \definecolor{ansi-yellow}{HTML}{DDB62B}
    \definecolor{ansi-yellow-intense}{HTML}{B27D12}
    \definecolor{ansi-blue}{HTML}{208FFB}
    \definecolor{ansi-blue-intense}{HTML}{0065CA}
    \definecolor{ansi-magenta}{HTML}{D160C4}
    \definecolor{ansi-magenta-intense}{HTML}{A03196}
    \definecolor{ansi-cyan}{HTML}{60C6C8}
    \definecolor{ansi-cyan-intense}{HTML}{258F8F}
    \definecolor{ansi-white}{HTML}{C5C1B4}
    \definecolor{ansi-white-intense}{HTML}{A1A6B2}
    \definecolor{ansi-default-inverse-fg}{HTML}{FFFFFF}
    \definecolor{ansi-default-inverse-bg}{HTML}{000000}

    % common color for the border for error outputs.
    \definecolor{outerrorbackground}{HTML}{FFDFDF}

    % commands and environments needed by pandoc snippets
    % extracted from the output of `pandoc -s`
    \providecommand{\tightlist}{%
      \setlength{\itemsep}{0pt}\setlength{\parskip}{0pt}}
    \DefineVerbatimEnvironment{Highlighting}{Verbatim}{commandchars=\\\{\}}
    % Add ',fontsize=\small' for more characters per line
    \newenvironment{Shaded}{}{}
    \newcommand{\KeywordTok}[1]{\textcolor[rgb]{0.00,0.44,0.13}{\textbf{{#1}}}}
    \newcommand{\DataTypeTok}[1]{\textcolor[rgb]{0.56,0.13,0.00}{{#1}}}
    \newcommand{\DecValTok}[1]{\textcolor[rgb]{0.25,0.63,0.44}{{#1}}}
    \newcommand{\BaseNTok}[1]{\textcolor[rgb]{0.25,0.63,0.44}{{#1}}}
    \newcommand{\FloatTok}[1]{\textcolor[rgb]{0.25,0.63,0.44}{{#1}}}
    \newcommand{\CharTok}[1]{\textcolor[rgb]{0.25,0.44,0.63}{{#1}}}
    \newcommand{\StringTok}[1]{\textcolor[rgb]{0.25,0.44,0.63}{{#1}}}
    \newcommand{\CommentTok}[1]{\textcolor[rgb]{0.38,0.63,0.69}{\textit{{#1}}}}
    \newcommand{\OtherTok}[1]{\textcolor[rgb]{0.00,0.44,0.13}{{#1}}}
    \newcommand{\AlertTok}[1]{\textcolor[rgb]{1.00,0.00,0.00}{\textbf{{#1}}}}
    \newcommand{\FunctionTok}[1]{\textcolor[rgb]{0.02,0.16,0.49}{{#1}}}
    \newcommand{\RegionMarkerTok}[1]{{#1}}
    \newcommand{\ErrorTok}[1]{\textcolor[rgb]{1.00,0.00,0.00}{\textbf{{#1}}}}
    \newcommand{\NormalTok}[1]{{#1}}

    % Additional commands for more recent versions of Pandoc
    \newcommand{\ConstantTok}[1]{\textcolor[rgb]{0.53,0.00,0.00}{{#1}}}
    \newcommand{\SpecialCharTok}[1]{\textcolor[rgb]{0.25,0.44,0.63}{{#1}}}
    \newcommand{\VerbatimStringTok}[1]{\textcolor[rgb]{0.25,0.44,0.63}{{#1}}}
    \newcommand{\SpecialStringTok}[1]{\textcolor[rgb]{0.73,0.40,0.53}{{#1}}}
    \newcommand{\ImportTok}[1]{{#1}}
    \newcommand{\DocumentationTok}[1]{\textcolor[rgb]{0.73,0.13,0.13}{\textit{{#1}}}}
    \newcommand{\AnnotationTok}[1]{\textcolor[rgb]{0.38,0.63,0.69}{\textbf{\textit{{#1}}}}}
    \newcommand{\CommentVarTok}[1]{\textcolor[rgb]{0.38,0.63,0.69}{\textbf{\textit{{#1}}}}}
    \newcommand{\VariableTok}[1]{\textcolor[rgb]{0.10,0.09,0.49}{{#1}}}
    \newcommand{\ControlFlowTok}[1]{\textcolor[rgb]{0.00,0.44,0.13}{\textbf{{#1}}}}
    \newcommand{\OperatorTok}[1]{\textcolor[rgb]{0.40,0.40,0.40}{{#1}}}
    \newcommand{\BuiltInTok}[1]{{#1}}
    \newcommand{\ExtensionTok}[1]{{#1}}
    \newcommand{\PreprocessorTok}[1]{\textcolor[rgb]{0.74,0.48,0.00}{{#1}}}
    \newcommand{\AttributeTok}[1]{\textcolor[rgb]{0.49,0.56,0.16}{{#1}}}
    \newcommand{\InformationTok}[1]{\textcolor[rgb]{0.38,0.63,0.69}{\textbf{\textit{{#1}}}}}
    \newcommand{\WarningTok}[1]{\textcolor[rgb]{0.38,0.63,0.69}{\textbf{\textit{{#1}}}}}
    \makeatletter
    \newsavebox\pandoc@box
    \newcommand*\pandocbounded[1]{%
      \sbox\pandoc@box{#1}%
      % scaling factors for width and height
      \Gscale@div\@tempa\textheight{\dimexpr\ht\pandoc@box+\dp\pandoc@box\relax}%
      \Gscale@div\@tempb\linewidth{\wd\pandoc@box}%
      % select the smaller of both
      \ifdim\@tempb\p@<\@tempa\p@
        \let\@tempa\@tempb
      \fi
      % scaling accordingly (\@tempa < 1)
      \ifdim\@tempa\p@<\p@
        \scalebox{\@tempa}{\usebox\pandoc@box}%
      % scaling not needed, use as it is
      \else
        \usebox{\pandoc@box}%
      \fi
    }
    \makeatother

    % Define a nice break command that doesn't care if a line doesn't already
    % exist.
    \def\br{\hspace*{\fill} \\* }
    % Math Jax compatibility definitions
    \def\gt{>}
    \def\lt{<}
    \let\Oldtex\TeX
    \let\Oldlatex\LaTeX
    \renewcommand{\TeX}{\textrm{\Oldtex}}
    \renewcommand{\LaTeX}{\textrm{\Oldlatex}}
    % Document parameters
    % Document title
    \title{QSG Index Rebalance Analysis}
    
    
    
    
    
    
    
% Pygments definitions
\makeatletter
\def\PY@reset{\let\PY@it=\relax \let\PY@bf=\relax%
    \let\PY@ul=\relax \let\PY@tc=\relax%
    \let\PY@bc=\relax \let\PY@ff=\relax}
\def\PY@tok#1{\csname PY@tok@#1\endcsname}
\def\PY@toks#1+{\ifx\relax#1\empty\else%
    \PY@tok{#1}\expandafter\PY@toks\fi}
\def\PY@do#1{\PY@bc{\PY@tc{\PY@ul{%
    \PY@it{\PY@bf{\PY@ff{#1}}}}}}}
\def\PY#1#2{\PY@reset\PY@toks#1+\relax+\PY@do{#2}}

\@namedef{PY@tok@w}{\def\PY@tc##1{\textcolor[rgb]{0.73,0.73,0.73}{##1}}}
\@namedef{PY@tok@c}{\let\PY@it=\textit\def\PY@tc##1{\textcolor[rgb]{0.24,0.48,0.48}{##1}}}
\@namedef{PY@tok@cp}{\def\PY@tc##1{\textcolor[rgb]{0.61,0.40,0.00}{##1}}}
\@namedef{PY@tok@k}{\let\PY@bf=\textbf\def\PY@tc##1{\textcolor[rgb]{0.00,0.50,0.00}{##1}}}
\@namedef{PY@tok@kp}{\def\PY@tc##1{\textcolor[rgb]{0.00,0.50,0.00}{##1}}}
\@namedef{PY@tok@kt}{\def\PY@tc##1{\textcolor[rgb]{0.69,0.00,0.25}{##1}}}
\@namedef{PY@tok@o}{\def\PY@tc##1{\textcolor[rgb]{0.40,0.40,0.40}{##1}}}
\@namedef{PY@tok@ow}{\let\PY@bf=\textbf\def\PY@tc##1{\textcolor[rgb]{0.67,0.13,1.00}{##1}}}
\@namedef{PY@tok@nb}{\def\PY@tc##1{\textcolor[rgb]{0.00,0.50,0.00}{##1}}}
\@namedef{PY@tok@nf}{\def\PY@tc##1{\textcolor[rgb]{0.00,0.00,1.00}{##1}}}
\@namedef{PY@tok@nc}{\let\PY@bf=\textbf\def\PY@tc##1{\textcolor[rgb]{0.00,0.00,1.00}{##1}}}
\@namedef{PY@tok@nn}{\let\PY@bf=\textbf\def\PY@tc##1{\textcolor[rgb]{0.00,0.00,1.00}{##1}}}
\@namedef{PY@tok@ne}{\let\PY@bf=\textbf\def\PY@tc##1{\textcolor[rgb]{0.80,0.25,0.22}{##1}}}
\@namedef{PY@tok@nv}{\def\PY@tc##1{\textcolor[rgb]{0.10,0.09,0.49}{##1}}}
\@namedef{PY@tok@no}{\def\PY@tc##1{\textcolor[rgb]{0.53,0.00,0.00}{##1}}}
\@namedef{PY@tok@nl}{\def\PY@tc##1{\textcolor[rgb]{0.46,0.46,0.00}{##1}}}
\@namedef{PY@tok@ni}{\let\PY@bf=\textbf\def\PY@tc##1{\textcolor[rgb]{0.44,0.44,0.44}{##1}}}
\@namedef{PY@tok@na}{\def\PY@tc##1{\textcolor[rgb]{0.41,0.47,0.13}{##1}}}
\@namedef{PY@tok@nt}{\let\PY@bf=\textbf\def\PY@tc##1{\textcolor[rgb]{0.00,0.50,0.00}{##1}}}
\@namedef{PY@tok@nd}{\def\PY@tc##1{\textcolor[rgb]{0.67,0.13,1.00}{##1}}}
\@namedef{PY@tok@s}{\def\PY@tc##1{\textcolor[rgb]{0.73,0.13,0.13}{##1}}}
\@namedef{PY@tok@sd}{\let\PY@it=\textit\def\PY@tc##1{\textcolor[rgb]{0.73,0.13,0.13}{##1}}}
\@namedef{PY@tok@si}{\let\PY@bf=\textbf\def\PY@tc##1{\textcolor[rgb]{0.64,0.35,0.47}{##1}}}
\@namedef{PY@tok@se}{\let\PY@bf=\textbf\def\PY@tc##1{\textcolor[rgb]{0.67,0.36,0.12}{##1}}}
\@namedef{PY@tok@sr}{\def\PY@tc##1{\textcolor[rgb]{0.64,0.35,0.47}{##1}}}
\@namedef{PY@tok@ss}{\def\PY@tc##1{\textcolor[rgb]{0.10,0.09,0.49}{##1}}}
\@namedef{PY@tok@sx}{\def\PY@tc##1{\textcolor[rgb]{0.00,0.50,0.00}{##1}}}
\@namedef{PY@tok@m}{\def\PY@tc##1{\textcolor[rgb]{0.40,0.40,0.40}{##1}}}
\@namedef{PY@tok@gh}{\let\PY@bf=\textbf\def\PY@tc##1{\textcolor[rgb]{0.00,0.00,0.50}{##1}}}
\@namedef{PY@tok@gu}{\let\PY@bf=\textbf\def\PY@tc##1{\textcolor[rgb]{0.50,0.00,0.50}{##1}}}
\@namedef{PY@tok@gd}{\def\PY@tc##1{\textcolor[rgb]{0.63,0.00,0.00}{##1}}}
\@namedef{PY@tok@gi}{\def\PY@tc##1{\textcolor[rgb]{0.00,0.52,0.00}{##1}}}
\@namedef{PY@tok@gr}{\def\PY@tc##1{\textcolor[rgb]{0.89,0.00,0.00}{##1}}}
\@namedef{PY@tok@ge}{\let\PY@it=\textit}
\@namedef{PY@tok@gs}{\let\PY@bf=\textbf}
\@namedef{PY@tok@ges}{\let\PY@bf=\textbf\let\PY@it=\textit}
\@namedef{PY@tok@gp}{\let\PY@bf=\textbf\def\PY@tc##1{\textcolor[rgb]{0.00,0.00,0.50}{##1}}}
\@namedef{PY@tok@go}{\def\PY@tc##1{\textcolor[rgb]{0.44,0.44,0.44}{##1}}}
\@namedef{PY@tok@gt}{\def\PY@tc##1{\textcolor[rgb]{0.00,0.27,0.87}{##1}}}
\@namedef{PY@tok@err}{\def\PY@bc##1{{\setlength{\fboxsep}{\string -\fboxrule}\fcolorbox[rgb]{1.00,0.00,0.00}{1,1,1}{\strut ##1}}}}
\@namedef{PY@tok@kc}{\let\PY@bf=\textbf\def\PY@tc##1{\textcolor[rgb]{0.00,0.50,0.00}{##1}}}
\@namedef{PY@tok@kd}{\let\PY@bf=\textbf\def\PY@tc##1{\textcolor[rgb]{0.00,0.50,0.00}{##1}}}
\@namedef{PY@tok@kn}{\let\PY@bf=\textbf\def\PY@tc##1{\textcolor[rgb]{0.00,0.50,0.00}{##1}}}
\@namedef{PY@tok@kr}{\let\PY@bf=\textbf\def\PY@tc##1{\textcolor[rgb]{0.00,0.50,0.00}{##1}}}
\@namedef{PY@tok@bp}{\def\PY@tc##1{\textcolor[rgb]{0.00,0.50,0.00}{##1}}}
\@namedef{PY@tok@fm}{\def\PY@tc##1{\textcolor[rgb]{0.00,0.00,1.00}{##1}}}
\@namedef{PY@tok@vc}{\def\PY@tc##1{\textcolor[rgb]{0.10,0.09,0.49}{##1}}}
\@namedef{PY@tok@vg}{\def\PY@tc##1{\textcolor[rgb]{0.10,0.09,0.49}{##1}}}
\@namedef{PY@tok@vi}{\def\PY@tc##1{\textcolor[rgb]{0.10,0.09,0.49}{##1}}}
\@namedef{PY@tok@vm}{\def\PY@tc##1{\textcolor[rgb]{0.10,0.09,0.49}{##1}}}
\@namedef{PY@tok@sa}{\def\PY@tc##1{\textcolor[rgb]{0.73,0.13,0.13}{##1}}}
\@namedef{PY@tok@sb}{\def\PY@tc##1{\textcolor[rgb]{0.73,0.13,0.13}{##1}}}
\@namedef{PY@tok@sc}{\def\PY@tc##1{\textcolor[rgb]{0.73,0.13,0.13}{##1}}}
\@namedef{PY@tok@dl}{\def\PY@tc##1{\textcolor[rgb]{0.73,0.13,0.13}{##1}}}
\@namedef{PY@tok@s2}{\def\PY@tc##1{\textcolor[rgb]{0.73,0.13,0.13}{##1}}}
\@namedef{PY@tok@sh}{\def\PY@tc##1{\textcolor[rgb]{0.73,0.13,0.13}{##1}}}
\@namedef{PY@tok@s1}{\def\PY@tc##1{\textcolor[rgb]{0.73,0.13,0.13}{##1}}}
\@namedef{PY@tok@mb}{\def\PY@tc##1{\textcolor[rgb]{0.40,0.40,0.40}{##1}}}
\@namedef{PY@tok@mf}{\def\PY@tc##1{\textcolor[rgb]{0.40,0.40,0.40}{##1}}}
\@namedef{PY@tok@mh}{\def\PY@tc##1{\textcolor[rgb]{0.40,0.40,0.40}{##1}}}
\@namedef{PY@tok@mi}{\def\PY@tc##1{\textcolor[rgb]{0.40,0.40,0.40}{##1}}}
\@namedef{PY@tok@il}{\def\PY@tc##1{\textcolor[rgb]{0.40,0.40,0.40}{##1}}}
\@namedef{PY@tok@mo}{\def\PY@tc##1{\textcolor[rgb]{0.40,0.40,0.40}{##1}}}
\@namedef{PY@tok@ch}{\let\PY@it=\textit\def\PY@tc##1{\textcolor[rgb]{0.24,0.48,0.48}{##1}}}
\@namedef{PY@tok@cm}{\let\PY@it=\textit\def\PY@tc##1{\textcolor[rgb]{0.24,0.48,0.48}{##1}}}
\@namedef{PY@tok@cpf}{\let\PY@it=\textit\def\PY@tc##1{\textcolor[rgb]{0.24,0.48,0.48}{##1}}}
\@namedef{PY@tok@c1}{\let\PY@it=\textit\def\PY@tc##1{\textcolor[rgb]{0.24,0.48,0.48}{##1}}}
\@namedef{PY@tok@cs}{\let\PY@it=\textit\def\PY@tc##1{\textcolor[rgb]{0.24,0.48,0.48}{##1}}}

\def\PYZbs{\char`\\}
\def\PYZus{\char`\_}
\def\PYZob{\char`\{}
\def\PYZcb{\char`\}}
\def\PYZca{\char`\^}
\def\PYZam{\char`\&}
\def\PYZlt{\char`\<}
\def\PYZgt{\char`\>}
\def\PYZsh{\char`\#}
\def\PYZpc{\char`\%}
\def\PYZdl{\char`\$}
\def\PYZhy{\char`\-}
\def\PYZsq{\char`\'}
\def\PYZdq{\char`\"}
\def\PYZti{\char`\~}
% for compatibility with earlier versions
\def\PYZat{@}
\def\PYZlb{[}
\def\PYZrb{]}
\makeatother


    % For linebreaks inside Verbatim environment from package fancyvrb.
    \makeatletter
        \newbox\Wrappedcontinuationbox
        \newbox\Wrappedvisiblespacebox
        \newcommand*\Wrappedvisiblespace {\textcolor{red}{\textvisiblespace}}
        \newcommand*\Wrappedcontinuationsymbol {\textcolor{red}{\llap{\tiny$\m@th\hookrightarrow$}}}
        \newcommand*\Wrappedcontinuationindent {3ex }
        \newcommand*\Wrappedafterbreak {\kern\Wrappedcontinuationindent\copy\Wrappedcontinuationbox}
        % Take advantage of the already applied Pygments mark-up to insert
        % potential linebreaks for TeX processing.
        %        {, <, #, %, $, ' and ": go to next line.
        %        _, }, ^, &, >, - and ~: stay at end of broken line.
        % Use of \textquotesingle for straight quote.
        \newcommand*\Wrappedbreaksatspecials {%
            \def\PYGZus{\discretionary{\char`\_}{\Wrappedafterbreak}{\char`\_}}%
            \def\PYGZob{\discretionary{}{\Wrappedafterbreak\char`\{}{\char`\{}}%
            \def\PYGZcb{\discretionary{\char`\}}{\Wrappedafterbreak}{\char`\}}}%
            \def\PYGZca{\discretionary{\char`\^}{\Wrappedafterbreak}{\char`\^}}%
            \def\PYGZam{\discretionary{\char`\&}{\Wrappedafterbreak}{\char`\&}}%
            \def\PYGZlt{\discretionary{}{\Wrappedafterbreak\char`\<}{\char`\<}}%
            \def\PYGZgt{\discretionary{\char`\>}{\Wrappedafterbreak}{\char`\>}}%
            \def\PYGZsh{\discretionary{}{\Wrappedafterbreak\char`\#}{\char`\#}}%
            \def\PYGZpc{\discretionary{}{\Wrappedafterbreak\char`\%}{\char`\%}}%
            \def\PYGZdl{\discretionary{}{\Wrappedafterbreak\char`\$}{\char`\$}}%
            \def\PYGZhy{\discretionary{\char`\-}{\Wrappedafterbreak}{\char`\-}}%
            \def\PYGZsq{\discretionary{}{\Wrappedafterbreak\textquotesingle}{\textquotesingle}}%
            \def\PYGZdq{\discretionary{}{\Wrappedafterbreak\char`\"}{\char`\"}}%
            \def\PYGZti{\discretionary{\char`\~}{\Wrappedafterbreak}{\char`\~}}%
        }
        % Some characters . , ; ? ! / are not pygmentized.
        % This macro makes them "active" and they will insert potential linebreaks
        \newcommand*\Wrappedbreaksatpunct {%
            \lccode`\~`\.\lowercase{\def~}{\discretionary{\hbox{\char`\.}}{\Wrappedafterbreak}{\hbox{\char`\.}}}%
            \lccode`\~`\,\lowercase{\def~}{\discretionary{\hbox{\char`\,}}{\Wrappedafterbreak}{\hbox{\char`\,}}}%
            \lccode`\~`\;\lowercase{\def~}{\discretionary{\hbox{\char`\;}}{\Wrappedafterbreak}{\hbox{\char`\;}}}%
            \lccode`\~`\:\lowercase{\def~}{\discretionary{\hbox{\char`\:}}{\Wrappedafterbreak}{\hbox{\char`\:}}}%
            \lccode`\~`\?\lowercase{\def~}{\discretionary{\hbox{\char`\?}}{\Wrappedafterbreak}{\hbox{\char`\?}}}%
            \lccode`\~`\!\lowercase{\def~}{\discretionary{\hbox{\char`\!}}{\Wrappedafterbreak}{\hbox{\char`\!}}}%
            \lccode`\~`\/\lowercase{\def~}{\discretionary{\hbox{\char`\/}}{\Wrappedafterbreak}{\hbox{\char`\/}}}%
            \catcode`\.\active
            \catcode`\,\active
            \catcode`\;\active
            \catcode`\:\active
            \catcode`\?\active
            \catcode`\!\active
            \catcode`\/\active
            \lccode`\~`\~
        }
    \makeatother

    \let\OriginalVerbatim=\Verbatim
    \makeatletter
    \renewcommand{\Verbatim}[1][1]{%
        %\parskip\z@skip
        \sbox\Wrappedcontinuationbox {\Wrappedcontinuationsymbol}%
        \sbox\Wrappedvisiblespacebox {\FV@SetupFont\Wrappedvisiblespace}%
        \def\FancyVerbFormatLine ##1{\hsize\linewidth
            \vtop{\raggedright\hyphenpenalty\z@\exhyphenpenalty\z@
                \doublehyphendemerits\z@\finalhyphendemerits\z@
                \strut ##1\strut}%
        }%
        % If the linebreak is at a space, the latter will be displayed as visible
        % space at end of first line, and a continuation symbol starts next line.
        % Stretch/shrink are however usually zero for typewriter font.
        \def\FV@Space {%
            \nobreak\hskip\z@ plus\fontdimen3\font minus\fontdimen4\font
            \discretionary{\copy\Wrappedvisiblespacebox}{\Wrappedafterbreak}
            {\kern\fontdimen2\font}%
        }%

        % Allow breaks at special characters using \PYG... macros.
        \Wrappedbreaksatspecials
        % Breaks at punctuation characters . , ; ? ! and / need catcode=\active
        \OriginalVerbatim[#1,codes*=\Wrappedbreaksatpunct]%
    }
    \makeatother

    % Exact colors from NB
    \definecolor{incolor}{HTML}{303F9F}
    \definecolor{outcolor}{HTML}{D84315}
    \definecolor{cellborder}{HTML}{CFCFCF}
    \definecolor{cellbackground}{HTML}{F7F7F7}

    % prompt
    \makeatletter
    \newcommand{\boxspacing}{\kern\kvtcb@left@rule\kern\kvtcb@boxsep}
    \makeatother
    \newcommand{\prompt}[4]{
        {\ttfamily\llap{{\color{#2}[#3]:\hspace{3pt}#4}}\vspace{-\baselineskip}}
    }
    

    
    % Prevent overflowing lines due to hard-to-break entities
    \sloppy
    % Setup hyperref package
    \hypersetup{
      breaklinks=true,  % so long urls are correctly broken across lines
      colorlinks=true,
      urlcolor=urlcolor,
      linkcolor=linkcolor,
      citecolor=citecolor,
      }
    % Slightly bigger margins than the latex defaults
    
    \geometry{verbose,tmargin=1in,bmargin=1in,lmargin=1in,rmargin=1in}
    
    

\begin{document}
    
    \maketitle
    
    

    
    \section*{QSG Index Rebalance
Analysis}\label{qsg-index-rebalance-analysis}

\textbf{Repository:}
\url{https://github.com/Minouneshan/QSG-Index-Rebalance-Project}\\
\textbf{Contact:} Minoneshan@utexas.edu

\begin{center}\rule{0.5\linewidth}{0.5pt}\end{center}

    \subsection*{Table of Contents}\label{table-of-contents}

\begin{itemize}
\tightlist
\item
  Executive Summary
\item
  \begin{enumerate}
  \def\labelenumi{\arabic{enumi}.}
  \tightlist
  \item
    Introduction and Objective
  \end{enumerate}
\item
  \begin{enumerate}
  \def\labelenumi{\arabic{enumi}.}
  \setcounter{enumi}{1}
  \tightlist
  \item
    Data Sources and Preparation
  \end{enumerate}
\item
  \begin{enumerate}
  \def\labelenumi{\arabic{enumi}.}
  \setcounter{enumi}{2}
  \tightlist
  \item
    Strategy Design and Implementation
  \end{enumerate}
\item
  \begin{enumerate}
  \def\labelenumi{\arabic{enumi}.}
  \setcounter{enumi}{3}
  \tightlist
  \item
    Parameter Sweeps and Optimization
  \end{enumerate}
\item
  \begin{enumerate}
  \def\labelenumi{\arabic{enumi}.}
  \setcounter{enumi}{4}
  \tightlist
  \item
    Results: Performance Metrics and Analysis
  \end{enumerate}
\item
  \begin{enumerate}
  \def\labelenumi{\arabic{enumi}.}
  \setcounter{enumi}{5}
  \tightlist
  \item
    Risk Management and Portfolio Constraints
  \end{enumerate}
\item
  \begin{enumerate}
  \def\labelenumi{\arabic{enumi}.}
  \setcounter{enumi}{6}
  \tightlist
  \item
    Decision-Making Process and Assumptions
  \end{enumerate}
\item
  \begin{enumerate}
  \def\labelenumi{\arabic{enumi}.}
  \setcounter{enumi}{7}
  \tightlist
  \item
    Tools and Technical Implementation
  \end{enumerate}
\item
  \begin{enumerate}
  \def\labelenumi{\arabic{enumi}.}
  \setcounter{enumi}{8}
  \tightlist
  \item
    Conclusions and Recommendations
  \end{enumerate}
\end{itemize}

    \subsection*{Executive Summary}\label{executive-summary}

This report presents a robust, modular pipeline for backtesting and
optimizing index event trading strategies, focused on S\&P 400/500/600
additions. The best-performing strategy is Buy-and-Hold (entry lag=1,
hold\_days=5), delivering the highest Sharpe and net PnL. Hedged
Momentum strategies reduce risk but also lower returns. All results are
robust to missing data, liquidity constraints, and transaction/overnight
costs. Practical recommendations include focusing on momentum-based
approaches and refining mean reversion logic.

    \textbf{Quantitative Trader Candidate Project --- Quantitative
Strategies Group (QSG)}

This notebook documents the design, implementation, and evaluation of
systematic trading strategies for index rebalancing events, as required
by the assignment.\\
It covers data preparation, strategy logic, parameter sweeps, risk
management, and performance analysis, with all code and results fully
reproducible.

    \subsubsection*{1. Introduction and
Objective}\label{introduction-and-objective}

\textbf{Objective:}\\
Analyze index addition events for S\&P indices since May 2022 to
identify and backtest systematic trading opportunities.\\
Showcase thought process, technical skills, and decision-making in a
real-world trading context.

\textbf{Assignment Requirements Addressed:} - Use provided index event
data (S\&P 400/500/600) and end-of-day price data (Yahoo Finance). -
Design and backtest strategies: Post-Announcement Momentum, Event Day
Reversion, Buy-and-Hold, Hedged Momentum. - Explore holding periods,
risk management, and hedging. - Apply liquidity and execution
constraints. - Document all steps, code, and results.

    \subsubsection*{2. Data Sources and
Preparation}\label{data-sources-and-preparation}

\subsection*{2.1. Event Data}\label{event-data}

\begin{itemize}
\tightlist
\item
  Source: Provided Excel file with index addition events (S\&P
  400/500/600).
\item
  Key columns: \texttt{Index\ Change}, \texttt{Announced},
  \texttt{Trade\ Date}, \texttt{Ticker}.
\end{itemize}

\subsection*{2.2. Price Data}\label{price-data}

\begin{itemize}
\tightlist
\item
  Source: Yahoo Finance via \texttt{yfinance} Python package.
\item
  Data: Daily OHLCV for all tickers in the event file, plus SPY for
  hedging.
\item
  All prices are adjusted for splits/dividends.
\end{itemize}

\subsection*{2.3. Data Cleaning and Filtering}\label{data-cleaning-and-filtering}

\begin{itemize}
\tightlist
\item
  Exclude events with missing price data (delisted, illiquid, or ticker
  changes).
\item
  Only keep events where both ticker and event date exist in price data.
\item
  Apply liquidity filters: position size \textless= 1\% of 20-day ADV.
\item
  All trades executed at open/close prices only.
\end{itemize}

\textbf{Scheduled vs.~One-off Events}

The assignment suggests analyzing regularly scheduled index reviews
separately from one-off events. In this project, the analysis focused on
all S\&P 400/500/600 addition events as a group. Scheduled and one-off
events were not explicitly separated due to data limitations and the
relatively small sample size. Future work could explore performance
differences between these event types if more granular event
classification is available.

\textbf{Data Cleaning and Filtering:}\\
- Events with missing price data (e.g., delisted, illiquid, or ticker
changes) were excluded. - Only events where both ticker and event date
had available price data were included. - Liquidity filters: position
size \textless= 1\% of 20-day ADV. - All trades executed at open/close
prices only.

\textbf{Rationale:}\\
These exclusions ensure that results are not biased by survivorship or
illiquidity and that all backtests reflect realistic trading conditions.

\textbf{Note:}\\
- Price data was retrieved programmatically using \texttt{yfinance}. -
Exclusion stats (e.g., missing price, illiquid, too small) are tracked
and reported for transparency. - All trades are executed at open/close
prices, matching assignment requirements.

    
    \begin{Verbatim}[commandchars=\\\{\}]
   Announced Trade Date Index Change   Ticker Action  Last Px       Sector  \textbackslash{}
0 2024-07-23 2024-07-30      S\&P 600  SNDR US    Add    24.45  Industrials   
1 2024-07-23 2024-07-25      S\&P 400  AVTR US    Add    21.17   Healthcare   
2 2024-07-16 2024-07-19      S\&P 600  GTES US    Add    16.97  Industrials   
3 2024-06-27 2024-07-02      S\&P 600  PTGX US    Add    31.88   Healthcare   
4 2024-06-20 2024-06-27      S\&P 400  RYAN US    Add    55.36   Financials   

  Shs to Trade \$MM to Trade ADV to Trade  
0      5115943        125.1         6.53  
1     74808923       1583.7        10.48  
2     25154971        426.9         9.97  
3      7780892        248.1        11.67  
4     11252459        622.9        19.32  
    \end{Verbatim}

    
    \subsubsection*{3. Strategy Design and
Implementation}\label{strategy-design-and-implementation}

\subsection*{3.1. Post-Announcement
Momentum}\label{post-announcement-momentum}

\begin{itemize}
\tightlist
\item
  Buy stocks at the open the day after the index addition is announced.
\item
  Exit any day before or on the index ``trade date'' (parameter sweep
  for optimal holding period).
\end{itemize}

\subsection*{3.2. Event Day Reversion}\label{event-day-reversion}

\begin{itemize}
\tightlist
\item
  On the index ``trade date'', trade for mean reversion: short
  outperformers, buy underperformers vs.~SPY.
\item
  Exit after a parameterized holding period.
\end{itemize}

\subsection*{3.3. Buy-and-Hold}\label{buy-and-hold}

\begin{itemize}
\tightlist
\item
  Buy at the open after announcement, hold for a fixed period (parameter
  sweep for entry lag and holding days).
\end{itemize}

\subsection*{3.4. Hedged Momentum}\label{hedged-momentum}

\begin{itemize}
\tightlist
\item
  Buy event stock, short SPY in proportion (hedge ratio), hold for N
  days.
\item
  Parameters: hedge ratio, holding period, ADV filter.
\end{itemize}

\subsection*{3.5. Parameter Sweeps and
Optimization}\label{parameter-sweeps-and-optimization}

\begin{itemize}
\tightlist
\item
  All strategies use parameter sweeps (via \texttt{sweeps.py}) and
  ML-based optimization (\texttt{optimizer.py}) to select best Sharpe
  ratio.
\item
  Results are robust to missing data and type errors.
\end{itemize}

\subsection*{3.6. Risk Management and
Constraints}\label{risk-management-and-constraints}

\begin{itemize}
\tightlist
\item
  Portfolio size: \$5,000,000 (no compounding).
\item
  Position sizing: \textless= 1\% of 20-day ADV.
\item
  Transaction cost: \$0.01/share.
\item
  Overnight cost: Fed Funds + 1.5\% (long), Fed Funds + 1\% (short).
\item
  Execution: open/close prices only.
\item
  Hedging: SPY used for all relevant strategies.
\end{itemize}

\subsection*{3.7. Code Structure and
Organization}\label{code-structure-and-organization}

\begin{itemize}
\tightlist
\item
  \texttt{strategies.py}: Core strategy logic (modular, robust).
\item
  \texttt{sweeps.py}: Parameter sweeps (separated for clarity and
  reusability).
\item
  \texttt{optimizer.py}: ML and grid optimization (handles empty/missing
  data gracefully).
\item
  \texttt{backtest\_strategies.py}: Orchestrates data prep,
  optimization, backtesting, and reporting.
\item
  \texttt{utils.py}, \texttt{data\_prep.py}: Supporting utilities and
  data loading.
\item
  All code is well-commented, with clear function names and logical
  structure.
\end{itemize}

\subsubsection*{4. Parameter Sweeps and
Optimization}\label{parameter-sweeps-and-optimization-1}

\paragraph{\texorpdfstring{Why use
\texttt{sweeps.py}?}{Why use sweeps.py?}}\label{why-use-sweeps.py}

\begin{itemize}
\tightlist
\item
  Separating parameter sweeps into their own module (\texttt{sweeps.py})
  is a best practice:

  \begin{itemize}
  \tightlist
  \item
    Keeps strategy logic clean and focused.
  \item
    Allows easy reuse and extension of sweep logic.
  \item
    Enables robust error handling and JSON serialization for reporting.
  \end{itemize}
\item
  Wrappers are used in \texttt{backtest\_strategies.py} to adapt scalar
  parameters for ML optimization, ensuring flexibility and
  maintainability.
\end{itemize}

\paragraph{Backtest and Optimizer
Design}\label{backtest-and-optimizer-design}

\begin{itemize}
\tightlist
\item
  The pipeline is robust, modular, and easy to follow.
\item
  All parameter casting and filtering is explicit, reducing runtime
  errors.
\item
  Results and figures are saved in a structured way for easy reporting
  and reproducibility.
\end{itemize}

\section*{5. Results: Performance Metrics and Analysis}
\subsection*{5.1. Summary Table of Best Strategies}
\label{summary-table-of-best-strategies}

\begin{longtable}[]{@{}
  >{\raggedright\arraybackslash}p{(\linewidth - 10\tabcolsep) * \real{0.2523}}
  >{\raggedright\arraybackslash}p{(\linewidth - 10\tabcolsep) * \real{0.3271}}
  >{\raggedright\arraybackslash}p{(\linewidth - 10\tabcolsep) * \real{0.0748}}
  >{\raggedright\arraybackslash}p{(\linewidth - 10\tabcolsep) * \real{0.1402}}
  >{\raggedright\arraybackslash}p{(\linewidth - 10\tabcolsep) * \real{0.0748}}
  >{\raggedright\arraybackslash}p{(\linewidth - 10\tabcolsep) * \real{0.1308}}@{}}
\toprule\noalign{}
\begin{minipage}[b]{\linewidth}\raggedright
Strategy
\end{minipage} & \begin{minipage}[b]{\linewidth}\raggedright
Parameters
\end{minipage} & \begin{minipage}[b]{\linewidth}\raggedright
Sharpe
\end{minipage} & \begin{minipage}[b]{\linewidth}\raggedright
Total Net PnL
\end{minipage} & \begin{minipage}[b]{\linewidth}\raggedright
Trades
\end{minipage} & \begin{minipage}[b]{\linewidth}\raggedright
Max Drawdown
\end{minipage} \\
\midrule\noalign{}
\endhead
\bottomrule\noalign{}
\endlastfoot
Post-Announcement Mom. & hold\_days=2 & 1.96 & \$77,747 & 94 &
\$90,509 \\
Event Day Reversion & threshold=0.001, hold\_days=4 & -10.90 &
-\$598,159 & 78 & \$585,939 \\
Buy-and-Hold & entry\_lag=1, hold\_days=5 & 5.30 & \$242,803 & 63 &
\$29,224 \\
Hedged Momentum & hedge\_ratio=0.5, hold\_days=6, min\_adv=5 & 1.54 &
\$23,461 & 55 & \$6,696 \\
\end{longtable}

\textbf{Notes:} - Buy-and-Hold is the best performing model by Sharpe
and net PnL. - Event Day Reversion is highly negative and should be
reviewed or excluded. - All metrics are based on out-of-sample backtests
with full transaction and carry costs.

\textbf{Hedge Impact Analysis:}

A direct comparison of hedged vs.~unhedged strategies shows that hedging
with SPY reduces portfolio volatility and drawdowns but also lowers
total returns. The Buy-and-Hold strategy (unhedged) achieved the highest
Sharpe and net PnL, while Hedged Momentum provided risk reduction at the
cost of lower returns. This trade-off should be considered when
designing live trading strategies.

    \begin{Verbatim}[commandchars=\\\{\}]
Post-Announcement Momentum: \{'total\_net\_pnl': 77746.69332015992, 'avg\_return':
0.004468234016977075, 'win\_rate': 0.5531914893617021, 'avg\_holding\_days':
1.6702127659574468, 'std\_return': 0.03638880184721602, 'sharpe':
1.9597057974304422, 'sortino': 3.610062763344993, 'max\_drawdown':
90509.27341976167, 'num\_trades': 94, 'exclusions': \{'no\_price': 40, 'illiquid':
0, 'too\_small': 0\}\}
Event Day Reversion: \{'total\_net\_pnl': -598159.009766159, 'avg\_return':
-0.040001647214194094, 'win\_rate': 0.10256410256410256, 'sharpe':
-10.895761531743437, 'sortino': -10.718228310543411, 'max\_drawdown':
585938.8354297638, 'num\_trades': 78, 'exclusions': \{'no\_price': 56, 'illiquid':
0, 'too\_small': 0, 'missing\_spy': 0\}\}
Buy-and-Hold: \{'total\_net\_pnl': 242802.9708914185, 'avg\_return':
0.017506180948579295, 'win\_rate': 0.6349206349206349, 'sharpe':
5.301814833448095, 'sortino': 28.48449061258461, 'max\_drawdown':
29223.731627807618, 'num\_trades': 63, 'exclusions': \{'no\_price': 71, 'illiquid':
0, 'too\_small': 0\}\}
Hedged Momentum: \{'total\_net\_pnl': 23461.006244715747, 'avg\_return':
0.006366535108125524, 'win\_rate': 0.5454545454545454, 'sharpe':
1.540225386000927, 'sortino': 1.8026972797103862, 'max\_drawdown':
6696.1686861657745, 'num\_trades': 55, 'exclusions': \{'no\_price': 79, 'illiquid':
0, 'too\_small': 0, 'no\_spy': 0\}\}
    \end{Verbatim}

    \begin{center}
    \adjustimage{max size={0.9\linewidth}{0.9\paperheight}}{QSG Index Rebalance Analysis_files/QSG Index Rebalance Analysis_9_0.png}
    \end{center}
    { \hspace*{\fill} \\}
    
    \subsubsection*{6. Risk Management and Portfolio
Constraints}\label{risk-management-and-portfolio-constraints}

\begin{itemize}
\tightlist
\item
  \textbf{Portfolio Size:} \$5,000,000 (no compounding).
\item
  \textbf{Position Sizing:} \textless= 1\% of 20-day ADV.
\item
  \textbf{Transaction Cost:} \$0.01/share.
\item
  \textbf{Overnight Cost:} Fed Funds + 1.5\% (long), Fed Funds + 1\%
  (short).
\item
  \textbf{Execution:} Open/close prices only.
\item
  \textbf{Hedging:} SPY used for all relevant strategies.
\item
  \textbf{Liquidity and Exclusions:} All exclusions (no price, illiquid,
  too small) are tracked and reported.
\end{itemize}

\textbf{Risk Discussion:} - Strategies are robust to missing data and
liquidity constraints. - Hedging reduces market risk but may reduce
returns. - Event Day Reversion strategy performed poorly, highlighting
the risk of mean reversion trades around index events.

\subsubsection*{7. Decision-Making Process and
Assumptions}\label{decision-making-process-and-assumptions}

\begin{itemize}
\tightlist
\item
  \textbf{Data Handling:}

  \begin{itemize}
  \tightlist
  \item
    Excluded events with missing price data or insufficient liquidity.
  \item
    Only included trades where both ticker and event date had available
    price data.
  \end{itemize}
\item
  \textbf{Parameter Selection:}

  \begin{itemize}
  \tightlist
  \item
    Used parameter sweeps and ML optimization to select best Sharpe
    ratio for each strategy.
  \end{itemize}
\item
  \textbf{Strategy Focus:}

  \begin{itemize}
  \tightlist
  \item
    Focused on S\&P 400/500/600 index additions.
  \item
    Did not cover every event type; prioritized strategies with strong
    potential.
  \end{itemize}
\item
  \textbf{Additional Sources:}

  \begin{itemize}
  \tightlist
  \item
    Used Yahoo Finance for price data, FRED for Fed Funds rates.
  \end{itemize}
\item
  \textbf{Assumptions:}

  \begin{itemize}
  \tightlist
  \item
    All trades executed at open/close prices.
  \item
    No compounding; portfolio size fixed.
  \item
    Transaction and carry costs applied as specified.
  \end{itemize}
\end{itemize}

\subsubsection*{8. Tools and Technical
Implementation}\label{tools-and-technical-implementation}

\begin{itemize}
\tightlist
\item
  \textbf{Python}: Main language for all data processing, backtesting,
  and analysis.
\item
  \textbf{Pandas, NumPy}: Data manipulation and analysis.
\item
  \textbf{Matplotlib}: Plotting and visualization.
\item
  \textbf{yfinance}: Downloading historical price data.
\item
  \textbf{Jupyter Notebook}: Reporting and documentation.
\item
  \textbf{Custom Modules}:

  \begin{itemize}
  \tightlist
  \item
    \texttt{strategies.py}, \texttt{sweeps.py}, \texttt{optimizer.py},
    \texttt{backtest\_strategies.py}, \texttt{utils.py},
    \texttt{data\_prep.py}
  \end{itemize}
\item
  \textbf{Code Quality:}

  \begin{itemize}
  \tightlist
  \item
    Modular, well-documented, robust to missing data and type errors.
  \item
    Clear separation of concerns and logical structure.
  \end{itemize}
\end{itemize}

\subsubsection*{9. Conclusions and
Recommendations}\label{conclusions-and-recommendations}

\begin{itemize}
\tightlist
\item
  \textbf{Best Strategy:} Buy-and-Hold (entry lag=1, hold\_days=5)
  delivered the highest Sharpe and net PnL.
\item
  \textbf{Hedging:} Hedged Momentum provided risk reduction but lower
  returns.
\item
  \textbf{Event Day Reversion:} Performed poorly; mean reversion is not
  a reliable edge around index events in this sample.
\item
  \textbf{Robustness:} The pipeline is robust, modular, and easy to
  extend for new strategies or data.
\item
  \textbf{Next Steps:}

  \begin{itemize}
  \tightlist
  \item
    Explore alternative hedges (sector ETFs, cointegrated pairs).
  \item
    Refine mean reversion logic or focus on momentum-based approaches.
  \item
    Consider more granular liquidity and risk controls for live trading.
  \end{itemize}
\end{itemize}


    % Add a bibliography block to the postdoc
    
    
    
\end{document}
